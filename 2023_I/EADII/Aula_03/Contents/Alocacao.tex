\section{Alocação}

%==================================================================================
\begin{frame}
\frametitle{Alocação de memória}


Existem três tipos de alocação:

\begin{itemize}
\justifying
\item \textbf{Estático:} a variável é alocada uma única vez, antes do início da execução do programa, e permanece alocada durante toda a execução.
\item \textbf{Automático:} a variável é alocada sempre que a execução da programação inicia o bloco no qual ela é declarada, permanecendo alocada até que o bloco seja finalizado. O valor inicial é indeterminado mas sempre que a declaração da variável é executada, ela assume o valor da sua expressão de iniciação, se houver, ou um valor indeterminado, em caso contrário.
\item \textbf{Por comando:} a alocação ocorre em decorrência da execução de comandos próprios de alocação de memória.
\end{itemize}



\end{frame}





%==================================================================================
\begin{frame}
\frametitle{Alocação de memória}


As funções de gerenciamento de memória permitem alocar, realocar e liberar espaços
de memória, e são declaradas no cabeçalho \textbf{stdlib.h}.

\begin{itemize}

\item void *malloc($size\_t \ tam$):\linebreak
Aloca espaço de memória de tamanho igual a \textbf{tam bytes}. O conteúdo do espaço alocado é indeterminado.



\item void *realloc(void *ptr,$size\_t \ tam$):\linebreak
Desloca o espaço apontado por \textbf{ptr}, realocando seu conteúdo em um novo espaço de tamanho igual a \textbf{tam bytes}.


\item void *calloc(unisgned int num, $size\_t \ tam$):\linebreak
Aloca uma quantidade de memória igual a \textbf{tam bytes}.


\end{itemize}


\begin{block}{free()}
Comando necessário para liberar memória.
\end{block}



\end{frame}


%==================================================================================
\begin{frame}
\frametitle{Vazamento de memória (Memory leak)}




\begin{block}{Memory leak}
O vazamento de memória é caracterizado pela existência de espaço de memória alocado,
mas que não pode ser acessado. Quando um espaço de memória é alocado pelas
funções \textbf{malloc}, \textbf{calloc} e \textbf{realloc}, ele permanece alocado até o término do programa
ou até que seja explicitamente desalocado. 
\end{block}


\end{frame}
%==================================================================================
\subsection{Malloc}


\begin{frame}[fragile]
  \frametitle{Exemplos de alocação de memória}
  

  \begin{block}{Exemplo: malloc}
  \begin{lstlisting}
#include <stdio.h>
#include <stdlib.h>

int main()
{
    int *mat;
    int lin=3,col=3,i,j;

    mat=(int *)malloc(lin*col*sizeof(int));

    *(mat + col*0 + 0)=2;

    for(i=0;i<lin;i++){
        for(j=0;j<col;j++){
            printf("%d\n",*(mat + col*i + j));
        }
    }
    return EXIT_SUCCESS;
}

  \end{lstlisting}
  \end{block}
\end{frame}

%==================================================================================
\subsection{Calloc}


\begin{frame}[fragile]
  \frametitle{Exemplos de alocação de memória}
  

  \begin{block}{Exemplo: calloc}
  \begin{lstlisting}
#include <stdio.h>
#include <stdlib.h>

int main()
{
    int *mat;
    int lin=3,col=3,i,j;

    mat=(int *)calloc(lin*col,sizeof(int));

    *(mat + col*0 + 0)=2;

    for(i=0;i<lin;i++){
        for(j=0;j<col;j++){
            printf("%d\n",*(mat + col*i + j));
        }
    }
    return EXIT_SUCCESS;
}


  \end{lstlisting}
  \end{block}
\end{frame}
%==================================================================================
\subsection{Realloc}

\begin{frame}[fragile]
  \frametitle{Exemplos de alocação de memória}
  

  \begin{block}{Exemplo: realloc}
  \begin{lstlisting}
#include <stdio.h>
#include <stdlib.h>

int main()
{
    int *mat,*vet;
    int lin=3,col=3,i,j;

    mat=(int *)calloc(lin*col,sizeof(int));

    *(mat + col*0 + 0)=2;

    vet=(int *)realloc(lin*col,sizeof(int));
    vet=mat;

    for(i=0;i<lin;i++){
        for(j=0;j<col;j++){
            printf("%d\n",*(vet + col*i + j));
        }
    }
    return EXIT_SUCCESS;
}

  \end{lstlisting}
  \end{block}
\end{frame}

%==================================================================================
\subsection{Memory leak}

\begin{frame}[fragile]
  \frametitle{Vazamento de memória}
  

  \begin{block}{Exemplo: memory leak}
  \begin{lstlisting}
#include <stdio.h>
#include <stdlib.h>

int main()
{
    int *mat,*vet;
    int lin=512,col=512,i,j;
    int tecla=1;



    while(tecla!=0){
        mat=(int *)calloc(lin*col,sizeof(int));

        printf("Digite: %d\n",tecla);
        scanf("%d",&tecla);
        //free(mat);

    }

    return EXIT_SUCCESS;
}


  \end{lstlisting}
  \end{block}
\end{frame}